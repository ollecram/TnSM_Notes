%%%%%%%%%%%%%%%%%%%% book.tex %%%%%%%%%%%%%%%%%%%%%%%%%%%%%

\documentclass[english, 11pt, a4paper]{book}

% Some specific typographic conventions used in Griffiths I2QM   START 
\usepackage{mathtools}			% equation tag with [..] instead of (..)
\newtagform{brackets}{[}{]}		% equation tag with [..] instead of (..)
\usetagform{brackets}			% equation tag with [..] instead of (..)
% Some specific typographic conventions used in Griffiths I2QM   END 

%\usepackage[nomath]{lmodern}
\usepackage[T1]{fontenc}
%\usepackage[italian]{babel}
% The following changes the Chapter heading from 'Chapter' to 'Lecture'
%$\addto\captionsenglish{\renewcommand{\chaptername}{Lecture}}
%$%\usepackage{fancyhdr}
%$\newcommand\chap[1]{%
%$ \chapter*{#1}%
%$  \addcontentsline{toc}{chapter}{#1}}
%$\newcommand\sect[1]{%
%$  \section*{#1}%
%$  \addcontentsline{toc}{section}{#1}}

% The following for using the Euro symbol
\usepackage[utf8x]{inputenc}
\usepackage{lmodern, textcomp}
  
% choose options for [] as required from the list
% in the Reference Guide, Sect. 2.2

\usepackage{makeidx}         % allows index generation
\usepackage{graphicx}        % standard LaTeX graphics tool
\usepackage{subcaption}      % for subfigures environments 
                             % when including figure files
\usepackage{multicol}        % used for the two-column index
\usepackage[bottom]{footmisc}% places footnotes at page bottom
% etc.
% see the list of further useful packages
% in the Reference Guide, Sects. 2.3, 3.1-3.3
\usepackage[normalem]{ulem}

\usepackage[shortlabels]{enumitem}	% to be able to resume enumerated lists

\usepackage{amsmath}	% To be able to slash
\usepackage{bm}	        % To use bold greek letters in math mode with \bm{}
\usepackage{amsfonts}	% To be able to use \mathbb ... 
\usepackage{amssymb}	% To be able to use \nmid ... 
\usepackage{amsthm}		% \qed, \qedhere
\usepackage{slashed}	% any character (dirac)
\usepackage[title,toc,page]{appendix}

% See https://tex.stackexchange.com/questions/36524/how-to-put-a-framed-box-around-text-math-environment/36528
\usepackage{collectbox}	% To make box around formulas

% *** AFTER THIS LINE *** 
%     put \usepackage{} for shared packages kept under ~\Links\repos\git\LaTeX_Styles

% Physics package 
% https://tex.stackexchange.com/questions/38978/how-can-i-manually-install-a-latex-package-debian-ubuntu-linux
\usepackage[italicdiff]{/home/marcello/Links/repos/git/LaTeX_Styles/physics}	
% To put accents below letters
\usepackage{/home/marcello/Links/repos/git/LaTeX_Styles//accents}

% To control vertical white space above and below equations
% see https://tex.stackexchange.com/questions/69662/how-to-globally-change-the-spacing-around-equations
\expandafter\def\expandafter\normalsize\expandafter{%
    \normalsize
    \setlength\abovedisplayskip{16pt}
    \setlength\belowdisplayskip{16pt}
    \setlength\abovedisplayshortskip{16pt}
    \setlength\belowdisplayshortskip{16pt}
}

% FROM BOXED_TEXT_ETC.tex

% To write two equations side by side
\usepackage{multicol}

% To use PGF/TikZ https://tex.stackexchange.com/questions/3622/best-way-to-generate-nice-function-plots-in-latex
\usepackage{tikz}
\usetikzlibrary{datavisualization}
\usetikzlibrary{datavisualization.formats.functions}

% To create a placeholder paragraph with Latin text
\usepackage{lipsum}

% To create framed text boxes with custom defined styles 
%\usepackage[linewidth=1pt]{mdframed}
\usepackage[framemethod=TikZ]{mdframed}
\mdfdefinestyle{MyFrame}{%
    linecolor=brown,			% blue, orange, brown, ...
    outerlinewidth=1pt,
    roundcorner=10pt,
    innertopmargin=\baselineskip,
    innerbottommargin=\baselineskip,
    innerrightmargin=15pt,
    innerleftmargin=15pt,
    backgroundcolor=gray!5!white}

		%Use for creating boxed/framed parts of text with nice borders

% To use extra symbols like dagger and double dagger in numbering footnotes 
\usepackage{footmisc}

% Allows aligning numbers at decimal point within `tabular environment
%\usepackage{siunitx}
%\sisetup{
%  round-mode          = places, % Rounds numbers
%  round-precision     = 2, % to 2 places
%}

% Force chapter numbering to restart within each part
\makeatletter
%\@addtoreset{chapter}{part}
\makeatletter


\makeindex             % used for the subject index
                       % please use the style svind.ist with
                       % your makeindex program


%%%%%%%%%%%%%%%%%%%%%%%%%%%%%%%%%%%%%%%%%%%%%%%%%%%%%%%%%%%%%%%%%%%%%

\begin{document}

\newcommand{\umlaut}[1]{\"#1}
\newcommand{\quotes}[1]{``#1''}
\newcommand{\ovr}[1]{\overline{#1}}
\newcommand{\sfT}{$\mathsf{T}$}
\newcommand{\udT}{\rotatebox[origin=c]{180}{$\mathsf{T}$}}
\newcommand{\avg}[1]{\langle{#1}\rangle}

%Bold calligraphic letters 
\newcommand{\N}{\mathbb{N}}	% integers
\newcommand{\Z}{\mathbb{Z}}	% relative
\newcommand{\Q}{\mathbb{Q}}	% rationals
\newcommand{\R}{\mathbb{R}}	% reals
\newcommand{\C}{\mathbb{C}}	% complex
\newcommand{\F}{\mathbb{F}}	% generic field 1
\newcommand{\K}{\mathbb{K}}	% generic field 2
\newcommand{\V}{\mathbb{V}}	% Shankar's for vector space V

%Plain calligraphic letters 
\newcommand{\cC}{\mathcal{C}}    % space 1
\newcommand{\cF}{\mathcal{F}}    % space 2
\newcommand{\cH}{\mathcal{H}}    % Calligraphic H for Hilbert space
\newcommand{\cS}{\mathcal{S}}    % space 3, Flow of energy (e.g in electromagnetism)
\newcommand{\cT}{\mathcal{T}}    % space 4 

\newcommand{\cI}{\mathcal{I}}    % Moment of Inertia
\newcommand{\cU}{\mathcal{U}}    % sets 1
\newcommand{\cV}{\mathcal{V}}    % sets 2
\newcommand{\cW}{\mathcal{W}}    % sets 3
\newcommand{\cP}{\mathcal{P}}    % sets 4, Momentum density (e.g in electromagnetism) 
\newcommand{\cQ}{\mathcal{Q}}    % sets 5
\newcommand{\cR}{\mathcal{R}}    % sets 6

\newcommand{\cL}{\mathcal{L}}    % Lagrangian density
\newcommand{\cE}{\mathcal{E}}    % Energy density (e.g in electromagnetism)
\newcommand{\cY}{\mathcal{Y}}    % Y

% Quaternions
\newcommand{\Qt}{\mathbb{H}}	% Hamilton's quaternions ('\H' APPARENTLY defined elsewhere by LaTeX}
\newcommand{\qu}{\mathbf{1}}     % 1
\newcommand{\qi}{\mathbf{i}}     % i
\newcommand{\qj}{\mathbf{j}}     % j
\newcommand{\qk}{\mathbf{k}}     % k

% Fraktur (Gothic) font (e.g for algebras)
\newcommand{\frk}[1]{\mathfrak{#1}}  

% To show argument of the exponential function vertically, i.e., as a superscript 
\newcommand{\vexp}[1]{\,e^{#1}}

% To type an angle as a number of degrees like 45^\circ
\newcommand{\degree}[1]{{#1}^\circ}

% To create not-bold vectors with a hat or check accent 
\newcommand{\hatv}[1]{\hat{#1}}
\newcommand{\chkv}[1]{\check{#1}}

% To create boldface vectors with a hat or check accent 
\newcommand{\hatvb}[1]{\vb{\hat{#1}}}
\newcommand{\chkvb}[1]{\vb{\check{#1}}}

% To create boldface greek letters (e.g. for denoting vectors) 
\newcommand{\bmath}[1]{\bm{#1}}  				% SAME AS \bm{#1} - NOT WORTH USING 
\newcommand{\chkbm}[1]{\boldmath{\check{#1}}}	% bold-check
\newcommand{\hatbm}[1]{\boldmath{\hat{#1}}}		% bold-hat

% To create <x|, |x> and <x|y> with unit vectors inside
\newcommand{\ubra}[1]{\bra*{\vu{#1}}}
\newcommand{\uket}[1]{\ket*{\vu{#1}}}
\newcommand{\uip}[2]{\ip*{\vu{#1}}{\vu{#2}}}

% To put accents below letters. 
\newcommand{\ut}[1]{\underaccent{\tilde}{#1}}
\newcommand{\uh}[1]{\underaccent{\hat}{#1}}
\newcommand{\form}[1]{\uh{#1}}

% To create italic, bold, bolditalic text
\newcommand{\tit}[1]{\textit{#1}}
\newcommand{\tbf}[1]{\textbf{#1}}
\newcommand{\tbi}[1]{\textit{\textbf{#1}}}

% Latin Modern sans serif |OR| Helvetica (SELECT)
\newcommand{\textlmss}{\fontfamily{lmss}\selectfont}
\newcommand{\texthv}{\fontfamily{phv}\selectfont}

% Latin Modern sans serif |OR| Helvetica (USE, within OR outside MATH !)
\newcommand{\tlmss}[1]{\text{\textlmss{#1}}}
\newcommand{\thv}[1]{\text{\texthv{#1}}}

% To use \tlmss{T} symbol to denote transpose 
\newcommand{\transp}[1]{{#1}^{\tlmss{T}}}

% To use \dagger symbol to denote operator Adjoint
\newcommand{\Adj}[1]{{#1}^\dagger}

% To denote the Hermitian conjugate with a '+' superscript
\newcommand{\Hconj}[1]{{#1}^{+}}

% To use \tlmss{Ker}, \tlmss{Coker} and \tlmss{Img} to denote Kernel, Co-Kernel & Image 
\newcommand{\Ker}{\tlmss{Ker}\,}
\newcommand{\Coker}{\tlmss{Coker}\,}
\newcommand{\Img}{\tlmss{Im}\,}

% To use \tlmss{Alt} and \tlmss{alt} to denote alternation 
\newcommand{\Alt}{\tlmss{Alt}\,}
\newcommand{\alt}{\tlmss{alt}\,}

% To use \tlmss{Ann} to denote annulets 
\newcommand{\Ann}{\tlmss{Ann}\,}

% Misc abbreviations
\newcommand{\ora}[1]{\overrightarrow{#1}}

\DeclareRobustCommand{\rchi}{{\mathpalette\irchi\relax}}
\newcommand{\irchi}[2]{\raisebox{\depth}{$#1\chi$}} % inner command, used by \rchi

% See https://tex.stackexchange.com/questions/36524/how-to-put-a-framed-box-around-text-math-environment/36528
\makeatletter
\newcommand{\mybox}{%
    \collectbox{%
        \setlength{\fboxsep}{1pt}%
        \fbox{\BOXCONTENT}%
    }%
}
\makeatother

\author{Kerson Huang}
\title{Statistical Mechanics}
\maketitle

\frontmatter%%%%%%%%%%%%%%%%%%%%%%%%%%%%%%%%%%%%%%%%%%%%%%%%%%%%%%

%\include{dedic}

%\include{Plan}	
\setcounter{tocdepth}{1}	% Must appear BEFORE \tableofcontents!
\tableofcontents
%\addappheadtotoc

\mainmatter%%%%%%%%%%%%%%%%%%%%%%%%%%%%%%%%%%%%%%%%%%%%%%%%%%%%%%%
%\setcounter{chapter}{-1}	% To start with Chapter 0 !!  
%\input{../FANCYBOX}		% Example of boxed/framed parts of text with nice borders

\begin{flushright}
\tit{Quaggiù tutto vi appare regolato\\
dal sorgere e calare di una stella;\\
Ci scambiamo il pensiero con dei suoni,\\
movimenti del viso e delle mani;\\
Cerchiamo un'armonia che spesso è guerra,\\
ma io vorrei restarci sulla Terra!} 
\end{flushright} 
 \part{THERMODYNAMICS AND KINETIC THEORY}
 %\setcounter{chapter}{1}
\chapter{The laws of Thermodynamics}\label{ch:1}
  
 %\chapter{Some Applications of Thermodynamics}\label{ch:2}  
 %\chapter{The Problem of Kinetic Theory}\label{ch:3}  
 %\chapter{The Equilibrium State of a Dilute Gas}\label{ch:4}  
 %\chapter{Transport Phenomena}\label{ch:5}
 %\chapter{The Chapman-Enskog Method}\label{ch:6}  
 
 \part{STATISTICAL MECHANICS}
 %\chapter{Classical Statistical Mechanics}\label{ch:7}  
 %\chapter{Canonical Ensemble and Grand Canonical Ensemble}\label{ch:8}  
 %\chapter{Quantum Statistical Mechanics}\label{ch:9}  
 %\chapter{The Partition Function}\label{ch:10}
 %\chapter{Ideal Fermi Gas}\label{ch:11}
 %\chapter{Ideal Bose Gas}\label{ch:12} 
 %\chapter{Imperfect Gases at Low Temperatures}\label{ch:13} 
 %\chapter{Cluster Expansions}\label{ch:14} 
 %\chapter{Phase Transitions}\label{ch:15} 
 
 \part{SPECIAL TOPICS IN STATISTICAL MECHANICS}
 %\chapter{The Ising Model}\label{ch:16}  
 %\chapter{The Onsager Solution}\label{ch:17}  
 %\chapter{Liquid Helium}\label{ch:18}  
 %\chapter{Hard-Sphere Bose Gas}\label{ch:18}  

\appendixpage
\appendix
 %\chapter{N-Body System of Identical Particles}\label{ch:AppA}
 %\chapter{The Pseudopotential}\label{ch:AppB}
 %\chapter{The Theorems of Yang and Lee}\label{ch:AppC}

\backmatter%%%%%%%%%%%%%%%%%%%%%%%%%%%%%%%%%%%%%%%%%%%%%%%%%%%%%%%
%\include{referenc}
\printindex

%%%%%%%%%%%%%%%%%%%%%%%%%%%%%%%%%%%%%%%%%%%%%%%%%%%%%%%%%%%%%%%%%%%%%%

\end{document}





